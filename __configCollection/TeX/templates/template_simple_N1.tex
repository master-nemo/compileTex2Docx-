% template created by N from sample by NikitaDmitryukThesisMagistr
%\input{settings/preamble.tex} %% inlined -- preamble.tex --
\documentclass[
candidate, % document type
subf, % use and configure subfig package for nested figure numbering
times % use Times font as main
]{disser}

% Кодировка и язык
\usepackage[T2A]{fontenc} % поддержка кириллицы
\usepackage[utf8]{inputenc} % кодировка исходного текста
\usepackage[english,russian]{babel} % переключение языков

% Геометрия страницы и графика
\usepackage[left=3cm, right=1cm, top=2cm, bottom=2cm]{geometry} % поля страницы
\usepackage{graphicx} % подключение графики
\usepackage{pdfpages} % вставка pdf-страниц

% Таблицы
\usepackage{array} % расширенные возможности для работы с таблицами
\usepackage{tabularx} % автоматический подбор ширины столбцов
\usepackage{dcolumn} % выравнивание чисел по разделителю

% Математика
\usepackage{bm} % полужирное начертание для математических символов
\usepackage{amsmath} % дополнительные математические возможности
\usepackage{amssymb} % дополнительные математические символы

% Библиография и ссылки
\usepackage{cite} % поддержка цитирования
\usepackage{hyperref} % создание гиперссылок

% Прочее
\usepackage{color} % работа с цветом
\usepackage{epstopdf} % конвертация eps в pdf
\usepackage{multirow} % объединение ячеек таблиц по вертикали
\usepackage{afterpage} % вставка материала после текущей страницы
\usepackage[font={normal}]{caption} % настройка подписей к рисункам и таблицам
\usepackage[onehalfspacing]{setspace} % полуторный интервал
\usepackage{fancyhdr} % установка колонтитулов
\usepackage{listings} % поддержка вставки исходного кода

% Установка шрифта Times New Roman
\renewcommand{\rmdefault}{ftm}

% Создание нового типа столбца для выравнивания содержимого по центру
\newcommand{\PreserveBackslash}[1]{\let\temp=\\#1\let\\=\temp}
\newcolumntype{C}[1]{>{\PreserveBackslash\centering}p{#1}}

% Настройка стиля страницы
\pagestyle{fancy}      % Использование стиля "fancy" для оформления страниц
\fancyhf{}              % Очистка текущих значений колонтитулов
\fancyfoot[C]{\thepage} % Установка номера страницы в нижнем колонтитуле по центру
\renewcommand{\headrulewidth}{0pt} % Удаление разделительной линии в верхнем колонтитуле

\captionsetup{format=hang,labelsep=period} % Настройка подписей к изображениям и таблицам
% \let\vec=\mathbf % Использование полужирного начертания для векторов
% Установка глубины оглавления
\setcounter{tocdepth}{2}
\graphicspath{{images/}} % Указание папки для поиска изображений

\pagestyle{footcenter} % Установка стилей страницы и главы
\chapterpagestyle{footcenter}

% % #########################################################################

\begin{document}
	% % #########################################################################
	
	% Включение титульного листа (первая страница файла Title.pdf)
	%\includepdf[pages={1-1}]{extra/Title.pdf}
	
	% Подключение глав
	%\input{chapters/introduction.tex}
	%\input{chapters/chapter_1.tex}
	%\input{chapters/chapter_2.tex}
	%\input{chapters/conclusion.tex}
	% % #########################################################################
	
	\newpage
	\begin{center}
		\textbf{\large 1. hdr}
	\end{center}
	\refstepcounter{chapter}
	%\addcontentsline{toc}{chapter}{1. hdr4toc}
	
	\section {***}
	
	
	
	% % #########################################################################
	% Библиографический список
	%\input{chapters/bibliography.tex}
	% % --- bibliography.tex ---
	%\newpage
	%\addcontentsline{toc}{chapter}{СПИСОК ИСПОЛЬЗОВАННЫХ ИСТОЧНИКОВ} % это будет отображаться в содержании
	%\renewcommand{\bibsection}{\centering\textbf{\large СПИСОК ИСПОЛЬЗОВАННЫХ ИСТОЧНИКОВ}} % смена названия библиографии по умолчанию
	%\bibliographystyle{biblio/gost2008n}
	%\bibliography{biblio/bibliography} % папка biblio, файл biblio.bib
	% % ################## 
	
	% Приложение (может быть пустым)
	%\input{chapters/appendix.tex}
	
\end{document}
