% template created by N from sample by NikitaDmitryukThesisMagistr
%\input{settings/preamble.tex} %% inlined -- preamble.tex --
\documentclass[
subf, % use and configure subfig package for nested figure numbering
times % use Times font as main
]{article}
%]{disser}

% Кодировка и язык
\usepackage[T2A]{fontenc} % поддержка кириллицы
\usepackage[utf8]{inputenc} % кодировка исходного текста
\usepackage[english,main=russian]{babel} % переключение языков

% Геометрия страницы и графика
\usepackage[left=3cm, right=1cm, top=2cm, bottom=2cm]{geometry} % поля страницы
\usepackage{graphicx} % подключение графики
\usepackage{pdfpages} % вставка pdf-страниц

% Таблицы
\usepackage{array} % расширенные возможности для работы с таблицами
\usepackage{tabularx} % автоматический подбор ширины столбцов
\usepackage{dcolumn} % выравнивание чисел по разделителю

% Математика
\usepackage{bm} % полужирное начертание для математических символов
\usepackage{amsmath} % дополнительные математические возможности
\usepackage{amssymb} % дополнительные математические символы

% Библиография и ссылки
%\usepackage{cite} % поддержка цитирования	/ see bellow - changed to biblatex
\usepackage{hyperref} % создание гиперссылок

% Прочее
\usepackage{color} % работа с цветом
\usepackage{epstopdf} % конвертация eps в pdf
\usepackage{multirow} % объединение ячеек таблиц по вертикали
\usepackage{afterpage} % вставка материала после текущей страницы
\usepackage[font={normal}]{caption} % настройка подписей к рисункам и таблицам
\usepackage[onehalfspacing]{setspace} % полуторный интервал
\usepackage{fancyhdr} % установка колонтитулов
\usepackage{listings} % поддержка вставки исходного кода

% Установка шрифта Times New Roman
\renewcommand{\rmdefault}{ftm}

% Создание нового типа столбца для выравнивания содержимого по центру
\newcommand{\PreserveBackslash}[1]{\let\temp=\\#1\let\\=\temp}
\newcolumntype{C}[1]{>{\PreserveBackslash\centering}p{#1}}

% Настройка стиля страницы
\pagestyle{fancy}      % Использование стиля "fancy" для оформления страниц
\fancyhf{}              % Очистка текущих значений колонтитулов
\fancyfoot[C]{\thepage} % Установка номера страницы в нижнем колонтитуле по центру
\renewcommand{\headrulewidth}{0pt} % Удаление разделительной линии в верхнем колонтитуле

\captionsetup{format=hang,labelsep=period} % Настройка подписей к изображениям и таблицам
% \let\vec=\mathbf % Использование полужирного начертания для векторов
% Установка глубины оглавления
\setcounter{tocdepth}{2}
\graphicspath{{images/}} % Указание папки для поиска изображений

\usepackage{float}	% use \begin{figure}[H] to disable autoplace

%\pagestyle{footcenter} % Установка стилей страницы и главы

% --------- biblatex --------
\usepackage[
backend=biber,
style=gost-numeric,
%style=gost-footnote, % стиль цитирования и библиографии
language=auto, % получение языка из babel
%autolang=other, % многоязычная библиография
sorting=none,	% сортировка по появлению в документе
]{biblatex}
\addbibresource{sample.bib}


% % #########################################################################

\begin{document}
	
% Включение титульного листа (первая страница файла Title.pdf)
%\includepdf[pages={1-1}]{extra/Title.pdf}

\tableofcontents
%\newpage

% Подключение глав
%\input{chapters/introduction.tex}
%\input{chapters/chapter_1.tex}
%\input{chapters/conclusion.tex}
% % #########################################################################

\begin{center}
% \textbf{\large 1. hdr}
\section{***} 	
%\addcontentsline{toc}{chapter}{header 4 toc}
\end{center}
	

Using \texttt{biblatex} you can display bibliography divided into sections, 
depending of citation type. 's cite! The Einstein's journal paper \cite{einstein} and the Dirac's 
book \cite{dirac} are physics related items.  And \cite{knuthwebsite} too.


% ---------------------------------------------------------------------------- %
\medskip
% --------- bibliography name change help --------- 
%\addcontentsline{toc}{section}{Список литературы} 	% это будет отображаться в содержании
%\addcontentsline{toc}{section}{Список источников} 	% это будет отображаться в содержании
%\renewcommand{\refname}{Список литературы} % default
%\renewcommand{\refname}{Список литературы} % default
%\renewcommand{\refname}{Список источников}	% если надо изменить
% --------- bibliography name hack 4 pandoc name same as latex --------- 
\section*{Список литературы}
\addcontentsline{toc}{section}{Список литературы} 	% это будет отображаться в содержании
\renewcommand{\refname}{}	% to avoid double (remove it and * in section to use number)
% --------- bibliography --------- 
\printbibliography

% Приложение (может быть пустым)
%\input{chapters/appendix.tex}
	
\end{document}
